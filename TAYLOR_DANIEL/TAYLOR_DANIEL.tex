% ---------------------------- %
% Assignment Report Template
% 
% Heidi Christensen, 2020
% --------------------------   %


\documentclass[11pt,oneside]{article}
\usepackage[utf8]{inputenc}
\usepackage{graphicx}


\title{Experimental report for the 2021 COM1005 Assignment: The Rambler's Problem\footnote{https://github.com/ArchDan/COM1005}}
\author{Daniel Taylor}
\date{\today}


\begin{document}

\maketitle

\section{Description of my branch-and-bound implementation}
I created the necessary classes for b-n-b, then set about filling in the abstract classes.
Most work occurs in RamblersState, as RamblersSearch only really needed accessor methods to get important bits of imformation.
I created accessors for coordinates (coords) and the localCost that would be necessary
\begin{itemize}
\item GoalPredicate():
\item Takes searcher, finds the goal it is looking for. If this doesn't match the current coordinates, it returns false, otherwise true
\end{itemize}

\begin{itemize}
	\item GetSuccessors():
	\item Probably the most extensive method in this class
	\item gets the x and y of the current 
	\item Then finds all surrounding nodes and adds them to an ArrayList of nodes that are in bounds
	\item for every node in the surrounding nodes, calculate the cost of moving to it from the current node
	\item Add that as a state to successors, and then return that
\end{itemize}

\begin{itemize}
	\item sameState():
	\item Checks if the coordinates are the same, if they are, returns true
\end{itemize}
	
These are the major functions that were implemented, after that, all that came down to it was running some coordinates through it in RunRamblersBB.java

\section{Description of my A* implementation}

This is an example of a table in \LaTeX

\begin{table}[ht]
    \centering
    \begin{tabular}{|c|c|}
        System      & Measurement [\%] \\ \hline
        Basic       & 40\% \\
        Improved v1 & 42\% \\
        Improved v2 & 43\% \\
    \end{tabular}
    \caption{Example table}
    \label{tab:my_label}
\end{table}

\section{Assessing efficiency}

Figure~\ref{fig:panda} shows and example of including a picture and giving it a caption and a label that you can reference in the text.

\begin{figure}[ht]
\centering
  \includegraphics[scale=0.4]{Panda.png}
  \caption{This is Heidi's cat Panda.}
  \label{fig:panda}
 \end{figure} 
 
\subsection{Assessing the efficiency of my branch-and-bound search algorithm}

\subsection{Assessing the efficiency of my A* search algorithm}

\subsection{Comparing the two search strategies}

\section{Conclusions}



\end{document}
